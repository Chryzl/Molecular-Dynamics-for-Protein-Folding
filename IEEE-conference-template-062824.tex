\documentclass[conference]{IEEEtran}
\IEEEoverridecommandlockouts
%Template version as of 6/27/2024

\usepackage{cite}
\usepackage{amsmath,amssymb,amsfonts}
\usepackage{algorithmic}
\usepackage{graphicx}
\usepackage{textcomp}
\usepackage{xcolor}
\usepackage{orcidlink}
\hypersetup{
    colorlinks=false,
    pdfborder={0 0 0},
}

\newcommand{\TODO}[1]{\textbf{\textcolor{red}{TODO: #1}}}

\def\BibTeX{{\rm B\kern-.05em{\sc i\kern-.025em b}\kern-.08em
    T\kern-.1667em\lower.7ex\hbox{E}\kern-.125emX}}
\begin{document}

\title{
    Investigating Protein Folding Dynamics Using Molecular Dynamics Simulations-based Approaches
}

\author{\IEEEauthorblockN{Christian Nix\,\orcidlink{0009-0006-5718-3452}}
\IEEEauthorblockA{\textit{School of Computation, Information and Technology (CIT)} \\
\textit{Technical University of Munich}\\
Munich, Germany \\
christian.nix@tum.de}
}

\maketitle

\begin{abstract}
\TODO{Lorem ipsum dolor sit amet, consetetur sadipscing elitr, sed diam nonumy eirmod tempor invidunt ut labore et dolore magna aliquyam erat, sed diam voluptua. At vero eos et accusam et justo duo dolores et ea rebum. Stet clita kasd gubergren, no sea takimata sanctus est Lorem ipsum dolor sit amet. Lorem ipsum dolor sit amet, consetetur sadipscing elitr, sed diam nonumy eirmod tempor invidunt ut labore et dolore magna aliquyam erat, sed diam voluptua. At vero eos et accusam et justo duo dolores et ea rebum. Stet clita kasd gubergren, no sea takimata sanctus est Lorem ipsum dolor sit amet. Lorem ipsum dolor sit amet, consetetur sadipscing elitr, sed diam nonumy eirmod tempor invidunt ut labore et dolore magna aliquyam erat, sed diam voluptua. At vero eos et accusam et justo duo dolores et ea rebum. Stet clita kasd gubergren, no sea takimata sanctus est Lorem ipsum dolor sit amet. Duis autem vel eum iriure dolor in hendrerit in vulputate velit esse molestie consequat, vel illum dolore eu feugiat nulla facilisis at vero eros et accumsan et iusto odio dignissim qui blandit praesent luptatum zzril delenit augue duis dolore te feugait nulla facilisi.}
\end{abstract}

\begin{IEEEkeywords}
\TODO{Index words to be added in the end.}
\end{IEEEkeywords}

\section{Introduction}
\cite{key}

% Rough outline in bullets:
% - Protein backround in genreal:
%   - What are proteins
%   - Why are they important/in what ways do they impact our lives
%   - Protein seq -(Anfinsen's dogma)-> structure --> function
%   - This is why knowing the structure is important

% - Protein folding problem (really get into it):
%   - What is the protein folding problem
%   - How is protein strucutre (primary, secondary, tertiary, quaternary structure)
%   - What do we know about folding
%       - Primarily driven by thermodynamics (minimizing free energy) know this from e.g. de novo
%       - Entropy only seocondary, contrary to earlier beliefs
%       ==> We end up with a folding funnel (early studied mathemetically as random processes)
%   - Since AF, predicting final, lowest energy structure is mostly solved
%   - However, dynamics, folding pathways, binding, etc. cannot be predicted by AF
%       - Important for understanding function, since proteins are dynamic and change conformation during function

% - How protein folding is studied:
%   - Experimental methods (laser for fast folders, etc.) BUT expensive and limited to what we can (abstractly) measure
%   - Computational methods
%       - Physics-based (MD simulations)
%           - Pros: detailed, physics-based, can study dynamics (we can actually SEE the system under study)
%           - Cons: computationally expensive, limited timescales (can only study small proteins or short timescales), force-field accuracy, etc.
%   - How does MD simulate protein folding
%       - MD explores conformational space following the Boltzmann distribution (at equilibrium)
%       - Running either long enough, or many short simulations, or using enhanced sampling methods, etc. we can sample folding events
%       - From these, we can extract folding pathways, intermediates, kinetics, thermodynamics

% - Aim of this work:
%   - Literature review and light practical work on how to investigate protein folding dynamics using MD simulations-based approaches
%   - Scope of this work:
%       - Overview over how to study the folding process in terms of the free energy surface via collective variables (historic, and SOTA models, etc.)
%       - Using Markov State Models (MSMs) to model folding kinetics and pathways (which in turn are the basis for retrieving pathways, intermediates, etc.) 
%   - Applying the above to a model protein system to demonstrate the approaches and HOPEFULLY do some minor "novel" thing (i.e. change some parameters and see what result changes etc.) in this seminar paper for my University course

% - Collective variables (CVs) and free energy surfaces (FES):
%   - What are CVs
%   - Why get them:
%       - Allow to visualize high-dimensional free energy landscape in low dimensions --> interpretable, intuitive, etc.
%       - Basis for enhanced sampling methods (e.g. metadynamics)
%       - By biasing MDs along CVs, we can sample rare events (e.g. folding) more efficiently, accessing relevant conformations faster
%   ==> Thus CVs are an essential backbone for studying protein folding dynamics via MD simulations

%   - But what can CVs pratcically not solve: If we want deep insights, the projection onto low-dimensional CVs can lead to loss of important information (histeric problem in MD analysis, i.e., hiding important states/transitions). But also enhanced sampling methods are not the ultimate solution for deep insights and e.g. pathway identification including kinetics and rate constants
% Kind of first work with TICAs on MD simulations: https://doi.org/10.1063/1.3554380 (2011) --> They seem to be some of the first to apply TICA to MD simulations
% Here they show that TICA == specifica version of variational approach https://doi.org/10.1063/1.4811489 (2013)
% Why TICA in the first place and genrally avout collection variables: https://doi.org/10.1016/j.sbi.2017.02.006 

% Free-Energy Landscape for ß Hairpin Folding from Combined Parallel Tempering and Metadynamics | Giovanni Bussi,* Francesco Luigi Gervasio,* Alessandro Laio,† and Michele Parrinello ==> CVs are used to drive metadynamics simulations but they have problems if they hide slow degrees of freedom, so they try to fix that by merging parallel tempering
% CVs go back FAR: Statistical mechanics of isomerization dynamics in liquids and the transition state approximation | Chandler 1978, isomerization but they do their math along a reaction coordinate q
% ML CVs: https://arxiv.org/pdf/2107.03943 (2021) --> Also very interesting, they mention many works. ==> Improtnatly, CVs need to be classified into whether they are applied after the simulation (i.e., for analysis, visualization, etc.) or during the simulation (i.e., for enhanced sampling)
% Review on CVs: Collective variables for the study of long-time kinetics from molecular trajectories: Theory and methods | Frank Noé --> rather "new" (2017)
% --> They find that a variational approach to solve the eifenvalue problem of the transfer operator is a more general form compared to how it is done in MSMs or TICA (not meaning that they become obsolete, but that they are special cases of this more general approach)

% Potential Project:
% - A general concern with TICA and the variational approach is computational efficiency for large systems and long trajectories. The computational effort scales with N^2 for N input coordinates, which becomes intractable if huge basis sets are used, such as the set of distances between all residues or atoms of a protein. The recently proposed hierarchical TICA method can obtain a coarse-grained yet accurate solution of the full TICA problem more efficiently [60], but there is still room for improvements in this area. (from Noe review)
% - Or what if we apply the formalism found here to the problem of training NNs. From ChatGPT: Use stochastic gradient MCMC (SGLD / SGHMC) to sample: If you want a stationary distribution 𝑝 ( 𝜃 ) ∝ 𝑒 − 𝛽 𝐿 ( 𝜃 ) p(θ)∝e −βL(θ) over parameters, use SGLD: these add calibrated Gaussian noise to gradient updates and have a known stationary distribution (under assumptions). They are exactly the analogue of Langevin / Hamiltonian dynamics in parameter space. Classic refs: Welling & Teh (SGLD) and SGHMC work. (maybe this is (koopman training) related: https://arxiv.org/pdf/2006.02361) IMPORTANT for applying TICA: "We also assume that the dynamics are statistically reversible, i.e., that the molecular system is simulated in thermal equilibrium." from https://doi.org/10.1063/1.4811489 --> Check if this is given

\section*{Acknowledgment}

\TODO{Acknowledgments to be added in the end.}

\bibliographystyle{IEEEtran}
\bibliography{references}

\end{document}
